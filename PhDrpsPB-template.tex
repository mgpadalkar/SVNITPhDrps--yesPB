%------------------------------------------------------------------------------
% A template for using the SVNITPhDrpsPB.cls (v 14.03.20) class file. This is based on the
% template used by Reema Patel.
%------------------------------------------------------------------------------
% author : Milind Padalkar (milind.padalkar@gmail.com)
% motivation: All my friends in M.Tech. & PhD
% first created : Created as SVNITPhDReport.cls sometime in 2010.
% last modified on : 20-03-2014
% version : 14.03.20
%------------------------------------------------------------------------------
% This template is provided without any warranty and should be modified by the
% user as per requirement.
%------------------------------------------------------------------------------
% To include the color for hyperlinks set the line 96 in SVNITPhDrpsPB.cls as 
% "colorlinks=true," from "colorlinks=false,".
%------------------------------------------------------------------------------
% Strictly compile using: LaTeX=>PS=>PDF
%------------------------------------------------------------------------------

\documentclass{SVNITPhDrpsPB}

% Necessary packages (Do not comment these)
\usepackage{fancyhdr}
\usepackage{longtable}
\usepackage{array}

%Preamble

%Do not omit the following fields
\rpsNumber{Third} %RPS number
\thesisType{Progress Seminar} %Type (Ex. Thesis)
\atitle{Mr.} %Author title. Valid entries : Mr. Ms. Mrs.
\author{My Name} %Author name
\title{My RPS report title} %Report title
\dept{Department Name} %Department name
\regno{DYYCOXXX} %Registration number 
\semester{Odd} %Even or odd
\stitleI{Dr.} %Supervisor title. Examples : Dr., Prof.
\supervisorI{Supervisor's Name} %Supervisor name
\supervisorA{NIT -- Surat} %Supervisor's affiliation / Institution
\addressInstN{Sardar Vallabhbhai} %Institute name line1
\addressInstI{S. V.} %Institute name's initials
\addressInstD{National Institute of Technology} %Institute name line2
\addressInstP{Surat} %Institute name line3 (Place)
\acMonth{December} %Month
\acYear{2013 -- 2014} %Academic Year
\calYear{2013} %Calander Year
\instlogo{./Figures/SVNITlogo} %Institute logo file path

% User packages (You can add as many as you want)
\usepackage{amssymb}  %  for double struck characters
\usepackage{subfigure} % for subfigures

\usepackage{amsmath}

% User commands
\newcolumntype{L}[1]{>{\raggedright\let\newline\\\arraybackslash\hspace{0pt}}m{#1}}
\newcolumntype{C}[1]{>{\centering\let\newline\\\arraybackslash\hspace{0pt}}m{#1}}
\newcolumntype{R}[1]{>{\raggedleft\let\newline\\\arraybackslash\hspace{0pt}}m{#1}}

\newcommand{\BigO}[1]{\ensuremath{\operatorname{O}\bigl(#1\bigr)}}


%\usepackage{enumitem}
\begin{document}

\addpageborder

%%Create the title page
\maketitle

\pagenumbering{roman}
\setcounter{page}{2}

%Create the Table1 (Admin) page
\putTableAdmin{\begin{small}
\begin{longtable}{|c|c|c|c|c|c|c|c|}

	%Row1
	\hline
	\textbf{Sr} &
	\textbf{Type} &
	\textbf{Sequence} &
	\textbf{Title of the Seminar} &
	\multicolumn{3}{|c|}{\textbf{Evaluation Committee}} &
	\textbf{Date} \\
	%Row2
	\textbf{No} &
	\textbf{} &
	\textbf{Number} &
	\textbf{} &
	\multicolumn{3}{|c|}{} &
	\textbf{} \\ \hline
	%Row3
	\textbf{} &
	\textbf{} &
	\textbf{} &
	\textbf{} &
	\textbf{Chairman} &
	\textbf{Supervisor} &
	\textbf{Examiner(s)} &
	\textbf{} \\ \hline
	%Row4 onwards
	%%%%%%%%%%%%%%%%%%%%%%%%%%%%%%%%%%%%%%%%%%%%%%%%%%
	%%% User Data to be added here                 %%%
	%%% Uncomment the code below to see an example %%%
	%%%%%%%%%%%%%%%%%%%%%%%%%%%%%%%%%%%%%%%%%%%%%%%%%%
	% Row 1
	1. &										   %%%
	CR &									  	   %%%
	First &										   %%%
	Title  &							     	   %%%
	Dr. Chairman Name&						 	   %%%
	Dr. Supervisor Name&					 	   %%%
	Dr. Examiner Name&						  	   %%%
	Date \\ \hline										   %%%
% Row 2
	1. &										   %%%
	CR &									  	   %%%
	First &										   %%%
	Title  &							     	   %%%
	Dr. Chairman Name&						 	   %%%
	Dr. Supervisor Name&					 	   %%%
	Dr. Examiner Name&						  	   %%%
	Date \\ \hline										   %%%
% Row 3
	1. &										   %%%
	CR &									  	   %%%
	First &										   %%%
	Title  &							     	   %%%
	Dr. Chairman Name&						 	   %%%
	Dr. Supervisor Name&					 	   %%%
	Dr. Examiner Name&						  	   %%%
	Date \\ \hline										   %%%
% Row 4
	1. &										   %%%
	CR &									  	   %%%
	First &										   %%%
	Title  &							     	   %%%
	Dr. Chairman Name&						 	   %%%
	Dr. Supervisor Name&					 	   %%%
	Dr. Examiner Name&						  	   %%%
	Date \\ \hline										   %%%
% Row 5
	1. &										   %%%
	CR &									  	   %%%
	First &										   %%%
	Title  &							     	   %%%
	Dr. Chairman Name&						 	   %%%
	Dr. Supervisor Name&					 	   %%%
	Dr. Examiner Name&						  	   %%%
	Date \\ \hline										   %%%
% Row 6
	1. &										   %%%
	CR &									  	   %%%
	First &										   %%%
	Title  &							     	   %%%
	Dr. Chairman Name&						 	   %%%
	Dr. Supervisor Name&					 	   %%%
	Dr. Examiner Name&						  	   %%%
	Date \\ \hline										   %%%
% Row 7
	1. &										   %%%
	CR &									  	   %%%
	First &										   %%%
	Title  &							     	   %%%
	Dr. Chairman Name&						 	   %%%
	Dr. Supervisor Name&					 	   %%%
	Dr. Examiner Name&						  	   %%%
	Date \\ \hline										   %%%
% Row 8
	1. &										   %%%
	CR &									  	   %%%
	First &										   %%%
	Title  &							     	   %%%
	Dr. Chairman Name&						 	   %%%
	Dr. Supervisor Name&					 	   %%%
	Dr. Examiner Name&						  	   %%%
	Date \\ \hline										   %%%
% Row 9
	1. &										   %%%
	CR &									  	   %%%
	First &										   %%%
	Title  &							     	   %%%
	Dr. Chairman Name&						 	   %%%
	Dr. Supervisor Name&					 	   %%%
	Dr. Examiner Name&						  	   %%%
	Date \\ \hline										   %%%								
\end{longtable}
\end{small}}

%Create the Table2 (Technical) page
\putTableTech{\begin{longtable}{| c | c | L{11cm} |}
	%Row1
	\hline
	\textbf{Sr} &
	\textbf{Seminar} &
	\textbf{Research Topics Covered, broadly} \\
	%Row2
	\textbf{No} &
	\textbf{Details} &
	\textbf{} \\ \hline
	%Row3
%	\textbf{} &
%	\textbf{} &
%	\textbf{} \\ \hline
	%Row4 onwards
	%%%%%%%%%%%%%%%%%%%%%%%%%%%%%%%%%%%%%%%%%%%%%%%%%%
	%%% User Data to be added here                 %%%
	%%% Uncomment the code below to see an example %%%
	%%%%%%%%%%%%%%%%%%%%%%%%%%%%%%%%%%%%%%%%%%%%%%%%%%
	%%%	1. &									   %%%
	%%%	First CR &								   %%%
	%%%	Research Topics Covered, broadly \\		   %%%
	%%%											   %%%
	%%%	%Row5									   %%%
	%%%	{} &									   %%%
	%%%	{} &									   %%%
	%%%	{} &									   %%%
	%%%	Domains in a Distributed setup &		   %%%
	%%%	Macwana &								   %%%
	%%%	Jinwala &								   %%%
	%%%	Zaveri &								   %%%
	%%%	2011 \\ \hline							   %%%
	%%%%%%%%%%%%%%%%%%%%%%%%%%%%%%%%%%%%%%%%%%%%%%%%%%
	1. &									   %%%
	First CR &								   %%%
	Title:\\
	%Row4									   %%%
	{} &									   %%%
	{} &									   %%%
	$\clubsuit$ Point 1 \\
	%Row5									   %%%
	{} &									   %%%
	{} &									   %%%
	$\clubsuit$ Point 2 \\ \hline

	1. &									   %%%
	First CR &								   %%%
	Title:\\
	%Row4									   %%%
	{} &									   %%%
	{} &									   %%%
	$\clubsuit$ Point 1 \\
	%Row5									   %%%
	{} &									   %%%
	{} &									   %%%
	$\clubsuit$ Point 2 \\ \hline
	
	1. &									   %%%
	First CR &								   %%%
	Title:\\
	%Row4									   %%%
	{} &									   %%%
	{} &									   %%%
	$\clubsuit$ Point 1 \\
	%Row5									   %%%
	{} &									   %%%
	{} &									   %%%
	$\clubsuit$ Point 2 \\ \hline
	
	1. &									   %%%
	First CR &								   %%%
	Title:\\
	%Row4									   %%%
	{} &									   %%%
	{} &									   %%%
	$\clubsuit$ Point 1 \\
	%Row5									   %%%
	{} &									   %%%
	{} &									   %%%
	$\clubsuit$ Point 2 \\ \hline
	
	1. &									   %%%
	First CR &								   %%%
	Title:\\
	%Row4									   %%%
	{} &									   %%%
	{} &									   %%%
	$\clubsuit$ Point 1 \\
	%Row5									   %%%
	{} &									   %%%
	{} &									   %%%
	$\clubsuit$ Point 2 \\ \hline
	
	1. &									   %%%
	First CR &								   %%%
	Title:\\
	%Row4									   %%%
	{} &									   %%%
	{} &									   %%%
	$\clubsuit$ Point 1 \\
	%Row5									   %%%
	{} &									   %%%
	{} &									   %%%
	$\clubsuit$ Point 2 \\ \hline
	
	1. &									   %%%
	First CR &								   %%%
	Title:\\
	%Row4									   %%%
	{} &									   %%%
	{} &									   %%%
	$\clubsuit$ Point 1 \\
	%Row5									   %%%
	{} &									   %%%
	{} &									   %%%
	$\clubsuit$ Point 2 \\ \hline
	
	1. &									   %%%
	First CR &								   %%%
	Title:\\
	%Row4									   %%%
	{} &									   %%%
	{} &									   %%%
	$\clubsuit$ Point 1 \\
	%Row5									   %%%
	{} &									   %%%
	{} &									   %%%
	$\clubsuit$ Point 2 \\ \hline
	
	1. &									   %%%
	First CR &								   %%%
	Title:\\
	%Row4									   %%%
	{} &									   %%%
	{} &									   %%%
	$\clubsuit$ Point 1 \\
	%Row5									   %%%
	{} &									   %%%
	{} &									   %%%
	$\clubsuit$ Point 2 \\ \hline
	
	1. &									   %%%
	First CR &								   %%%
	Title:\\
	%Row4									   %%%
	{} &									   %%%
	{} &									   %%%
	$\clubsuit$ Point 1 \\
	%Row5									   %%%
	{} &									   %%%
	{} &									   %%%
	$\clubsuit$ Point 2 \\ \hline
	
	1. &									   %%%
	First CR &								   %%%
	Title:\\
	%Row4									   %%%
	{} &									   %%%
	{} &									   %%%
	$\clubsuit$ Point 1 \\
	%Row5									   %%%
	{} &									   %%%
	{} &									   %%%
	$\clubsuit$ Point 2 \\ \hline
	
	1. &									   %%%
	First CR &								   %%%
	Title:\\
	%Row4									   %%%
	{} &									   %%%
	{} &									   %%%
	$\clubsuit$ Point 1 \\
	%Row5									   %%%
	{} &									   %%%
	{} &									   %%%
	$\clubsuit$ Point 2 \\ \hline											
\end{longtable}}

% Create the decleration page
\putdecleration

% Create the approval page
\putapproval

% Create the acknowledgement page
\putsvnitack{
My acknowledgment, My acknowledgment, My acknowledgment, My acknowledgment, My acknowledgment, My acknowledgment, My acknowledgment, My acknowledgment, My acknowledgment, My acknowledgment, My acknowledgment, My acknowledgment, My acknowledgment, My acknowledgment, My acknowledgment, My acknowledgment, My acknowledgment, My acknowledgment, My acknowledgment, My acknowledgment, My acknowledgment, My acknowledgment, My acknowledgment, My acknowledgment, My acknowledgment, My acknowledgment, My acknowledgment, My acknowledgment, My acknowledgment, My acknowledgment, My acknowledgment, My acknowledgment, My acknowledgment, My acknowledgment, My acknowledgment, My acknowledgment, My acknowledgment, My acknowledgment, My acknowledgment, My acknowledgment, My acknowledgment, My acknowledgment, My acknowledgment, My acknowledgment, My acknowledgment, My acknowledgment, My acknowledgment, My acknowledgment, My acknowledgment, My acknowledgment, My acknowledgment, vMy acknowledgment, My acknowledgment, My acknowledgment, My acknowledgment, My acknowledgment, My acknowledgment, My acknowledgment, My acknowledgment, My acknowledgment, My acknowledgment, My acknowledgment, My acknowledgment, My acknowledgment, My acknowledgment, My acknowledgment, My acknowledgment, My acknowledgment, My acknowledgment, My acknowledgment, My acknowledgment, My acknowledgment, My acknowledgment, My acknowledgment, My acknowledgment, My acknowledgment, My acknowledgment, My acknowledgment, My acknowledgment, My acknowledgment, My acknowledgment, My acknowledgment, My acknowledgment, My acknowledgment, My acknowledgment, My acknowledgment, My acknowledgment, My acknowledgment, My acknowledgment, My acknowledgment, My acknowledgment, My acknowledgment, My acknowledgment, My acknowledgment, My acknowledgment, My acknowledgment, My acknowledgment, My acknowledgment, My acknowledgment, My acknowledgment, My acknowledgment, My acknowledgment, My acknowledgment, My acknowledgment, My acknowledgment, My acknowledgment, My acknowledgment, My acknowledgment, My acknowledgment, My acknowledgment, My acknowledgment, My acknowledgment, My acknowledgment, My acknowledgment, My acknowledgment, My acknowledgment, My acknowledgment, My acknowledgment, My acknowledgment, My acknowledgment, My acknowledgment, My acknowledgment, My acknowledgment, My acknowledgment, My acknowledgment, My acknowledgment, vMy acknowledgment, My acknowledgment, My acknowledgment, My acknowledgment, My acknowledgment, My acknowledgment, My acknowledgment, My acknowledgment, My acknowledgment, My acknowledgment, My acknowledgment, My acknowledgment, My acknowledgment, My acknowledgment, My acknowledgment, My acknowledgment, My acknowledgment, My acknowledgment, My acknowledgment, My acknowledgment, My acknowledgment, My acknowledgment, My acknowledgment, My acknowledgment, My acknowledgment, My acknowledgment, My acknowledgment, My acknowledgment, My acknowledgment, My acknowledgment, My acknowledgment, My acknowledgment, My acknowledgment, My acknowledgment, My acknowledgment, My acknowledgment, My acknowledgment, My acknowledgment, My acknowledgment.
}

% List of publications
\label{page:pub}
\begin{center}
\textbf{\large Publications}
\end{center}
\vspace*{25pt}
\begin{enumerate}[{[1]}]

\item N. Surname, ``Publication title,'' \textit{Journal name}, 2013.

\item N. Surname, ``Publication title,'' \textit{Journal name}, 2013.

\item N. Surname, ``Publication title,'' \textit{Journal name}, 2013.

\item N. Surname, ``Publication title,'' \textit{Journal name}, 2013.

\item N. Surname, ``Publication title,'' \textit{Journal name}, 2013.

\item N. Surname, ``Publication title,'' \textit{Journal name}, 2013.

\item N. Surname, ``Publication title,'' \textit{Journal name}, 2013.

\item N. Surname, ``Publication title,'' \textit{Journal name}, 2013.

\item N. Surname, ``Publication title,'' \textit{Journal name}, 2013.

\item N. Surname, ``Publication title,'' \textit{Journal name}, 2013.

\item N. Surname, ``Publication title,'' \textit{Journal name}, 2013.

\item N. Surname, ``Publication title,'' \textit{Journal name}, 2013.

\item N. Surname, ``Publication title,'' \textit{Journal name}, 2013.

\end{enumerate}

\newpage

% Create the abstract page
\putsvnitabstract{
\label{sec:abstract}

My abstract, My abstract, My abstract, My abstract, My abstract, My abstract, My abstract, My abstract, My abstract, My abstract, My abstract, My abstract, My abstract, My abstract, My abstract, My abstract, My abstract, My abstract, My abstract, My abstract, My abstract, My abstract, My abstract, My abstract, My abstract, My abstract, My abstract, My abstract, My abstract, My abstract, My abstract, My abstract, My abstract, My abstract, My abstract, My abstract, My abstract, My abstract, My abstract, My abstract, My abstract, My abstract, My abstract, My abstract, My abstract, My abstract, My abstract, My abstract, My abstract, My abstract, My abstract, My abstract, My abstract, My abstract, My abstract, My abstract.

My abstract, My abstract, My abstract, My abstract, My abstract, My abstract, My abstract, My abstract, My abstract, My abstract, My abstract, My abstract, My abstract, My abstract, My abstract, My abstract, My abstract, My abstract, My abstract, My abstract, My abstract, My abstract, My abstract, My abstract, vvMy abstract, My abstract, My abstract, My abstract, My abstract, My abstract, My abstract, My abstract, My abstract, My abstract, My abstract, My abstract, My abstract, My abstract, My abstract, My abstract, My abstract, My abstract, My abstract, My abstract, My abstract, My abstract, My abstract, My abstract, My abstract, My abstract, My abstract, My abstract, My abstract, My abstract, My abstract, My abstract, My abstract, My abstract, My abstract, My abstract, My abstract, My abstract, My abstract, My abstract, My abstract, My abstract, My abstract.

My abstract, My abstract, My abstract, My abstract, My abstract, My abstract, My abstract, My abstract, My abstract, My abstract, My abstract, My abstract, My abstract, My abstract, My abstract, My abstract, My abstract, My abstract, My abstract, My abstract, My abstract, My abstract, My abstract, My abstract, My abstract, My abstract, My abstract, My abstract, My abstract, My abstract, My abstract, My abstract, My abstract, My abstract, My abstract, My abstract, My abstract, My abstract, My abstract, My abstract, My abstract, My abstract, My abstract, My abstract, My abstract, My abstract, My abstract, My abstract, My abstract, My abstract, My abstract, My abstract, My abstract, My abstract, My abstract, My abstract, My abstract, My abstract, My abstract, My abstract, My abstract, My abstract, My abstract, My abstract, My abstract, My abstract, My abstract, My abstract, My abstract, My abstract, My abstract, My abstract.
}

% Table of contents
\tableofcontents\newpage

% List of figures
%\addcontentsline{toc}{section}{List of Figures} % Entery in ToC
\listoffigures\newpage

% List of tables
%\addcontentsline{toc}{section}{List of Tables} % Entry in ToC
\listoftables\newpage

\clearpageborder

% To change page number and style at any time, use the following
%\pagenumbering{roman}
%\setcounter{page}{13}
\pagenumbering{arabic}
\setcounter{page}{1}

% Your RPS chapters

%%Chapter 1
\chapter{Introduction \& Motivation}
\input{./Chapters/ch1}

% Bibliography
\bibliographystyle{IEEEtran}
\bibliography{SVNITPhDbibtex}
\nocite{*}

\end{document}
